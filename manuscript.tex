\documentclass[extra,mreferee]{gji}
%\documentclass[extra,paper]{gji}
\usepackage{timet}
\usepackage{graphicx}

\title[Tesseroid forward modeling with adaptive discretization]{
Forward modeling of gravitational fields using tesseroids with adaptive
discretization
}

\author[Uieda et al.]{
Leonardo Uieda$^{1,2}$,
Val\'eria C. F. Barbosa$^{2}$,
Vanderlei C. Oliveira Jr$^{2}$
\\
$^1$Universidade do Estado do Rio de Janeiro, Rio de Janeiro, Brazil.
$^2$Observat\'orio Nacional, Rio de Janeiro, Brazil.
}

\begin{document}

\label{firstpage}
\maketitle

\begin{abstract}
\end{abstract}

%%%%%%%%%%%%%%%%%%%%%%%%%%%%%%%%%%%%%%%%%%%%%%%%%%%%%%%%%%%%%%%%%%%%%%%%%%%%%%
\section{Introduction}


We use the optimized formula of \citet{Grombein2013} with Cartesian integral
kernels.

The Cartesian kernels are faster to compute.

The Cartesian kernels don't have singularities in the poles.

The traditional spherical integral kernels suffer from a singularity at the
poles \citep{Heck2007, Wild-Pfeiffer2008}.

The Cartesian formula are numerically integrated using a Taylor series
expansion as proposed by \citet{Heck2007}.

The Taylor series expansion approach produces accurate results at low
latitudes.

The Taylor series expansion presents a decrease in accuracy towards the polar
regions.

The tesseroids have an approximately rectangular surface area at low latitudes
but collapse into an approximately triangular shape at the poles.

\citet{Grombein2013} use a near-zone separation to mitigate the increased error
in high latitudes.

In the so called "near-zone" of the computation point they use a finner
discretization with smaller tesseroids.

This is accomplished by dividing the tesseroids along the horizontal
dimensions.

The determination of an optimal size of the near-zone remains an open question
\citep{Grombein2013}.

We use the Gauss-Legendre Quadrature (GLQ) to numerically integrate the Cartesian
kernels instead of the Taylor series expansion \citep{Asgharzadeh2007}.

The GLQ integration consists of approximating the volume integral by a weighted sum of
the effect of point masses.

The point masses are distributed according to the roots of Legendre polynomials
\citep{Hildebrand1987}.

An advantage of the GLQ approach is that the accuracy of integration can be
controlled by the number of point masses used.

A disadvantage of the GLQ is the increased computation time as the number of
point masses increases.

There is a trade-off between accuracy and computation time.

\citet{Ku1977} suggests that the accuracy of the GLQ integration depends on
the ratio between distance to the computation point and the distance between
adjacent point masses.

\citet{Ku1977} proposes an empirical criteria that the distance between adjacent
point masses should be less than the distance to the computation point.

This empirical criteria was used by \citet{Asgharzadeh2007} as a way to
choose the number of point masses used in the GLQ integration.

They suggested using this criteria for gravitational attraction as well as
the gravity gradient tensor (or Marussi tensor) of a tesseroid.

\citet{Ku1977} presented results for the vertical component of the
gravitational acceleration ($g_z$) caused by a right rectangular prism.

To our knowledge, there has been no investigation if the empirical criteria of
\citet{Ku1977} is valid for the second derivatives of the
gravitational potential.

There has also been no attempt to quantify the error committed in the GLQ
integration of gravity gradients when applying the criteria of \citet{Ku1977}.

\citet{Li2011} devised an automatic algorithm to enforce the criteria of
\citet{Ku1977}.

Their algorithm keeps the number of points masses per tesseroid fixed.

They then check the ratio between the minimum distance to the computation point
and the largest dimension of the tesseroid.

If the ratio below a specified threshold value, the tesseroid is divided into
smaller tesseroids.

This division is repeated recursively until all tesseroids obey the criteria.

The GLQ integration is then performed for each of the smaller tesseroids
using the fixed number of point masses.

The algorithm of \citet{Li2011} is similar to the near-zone separation used by
\citet{Grombein2013}.

The advantage of the adaptive discretization of \citet{Li2011} over simply
increasing the number of points masses in the GLQ is that the
amount of point masses will be greater only close to the computation point.

This makes the adaptive discretization more computationally efficient.

We propose an improvement over the adaptive discretization algorithm of
\citet{Li2011} to further increase the computational efficiency.

We will also quantify the error committed for the gravity gradients.

This is done in an attempt to establish an optimal minimum threshold for the
ratio between the distance to the computation point and the dimensions of the
tesseroid.


%%%%%%%%%%%%%%%%%%%%%%%%%%%%%%%%%%%%%%%%%%%%%%%%%%%%%%%%%%%%%%%%%%%%%%%%%%%%%%
\section{Methodology}


%%%%%%%%%%%%%%%%%%%%%%%%%%%%%%%%%%%%%%%%%%%%%%%%%%%%%%%%%%%%%%%%%%%%%%%%%%%%%%
\section{Results}


%%%%%%%%%%%%%%%%%%%%%%%%%%%%%%%%%%%%%%%%%%%%%%%%%%%%%%%%%%%%%%%%%%%%%%%%%%%%%%
\section{Discussion}

%%%%%%%%%%%%%%%%%%%%%%%%%%%%%%%%%%%%%%%%%%%%%%%%%%%%%%%%%%%%%%%%%%%%%%%%%%%%%%
\section{Conclusions}

%%%%%%%%%%%%%%%%%%%%%%%%%%%%%%%%%%%%%%%%%%%%%%%%%%%%%%%%%%%%%%%%%%%%%%%%%%%%%%
\section{Acknowledgments}


\bibliographystyle{gji}
\bibliography{references.bib}
\end{document}
